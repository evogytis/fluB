\documentclass[stdletter,letterpaper,addrfromright,orderfromdateto,dateleft,11pt,noaddrto,sigleft]{newlfm}
\topmarginskip{-0.4in}
\bottommarginskip{-1.5in}
\leftmarginsize{1in}
\rightmarginsize{1.25in}
\sigskipbefore{0.2in}
\sigskipafter{0in}
\noLines
\nolines
\noHeadline
\noheadline
\signature{Gytis Dudas}

\namefrom{}
\addrfrom{Inst.\ of Evolutionary Biology \\ University of Edinburgh \\ Ashworth Laboratories \\ King's Buildings \\ Edinburgh, UK}
\emailfrom{g.dudas@sms.ed.ac.uk}

\greetto{Dear Editor,}
\closeline{Sincerely,}

% comments
\usepackage{color}
\usepackage{ulem}
\definecolor{purple}{rgb}{0.459,0.109,0.538}
\def\tb#1#2{\sout{#1} \textcolor{purple}{#2}}
\def\tbc#1{\textcolor{purple}{[#1]}}
 
\begin{document}

\begin{newlfm}
Please find, attached, our manuscript entitled ``Reassortment between influenza B lineages and the emergence of a co-adapted PB1-PB2-HA gene complex''.
We would be grateful if you considered it for publication in \textit{Molecular Biology and Evolution}.
The manuscript provides a comprehensive view of the reassortment history of influenza B viruses since the 1980s and identifies 3 segments exhibiting dynamics analogous to what is considered a prelude to sympatric speciation in multicellular taxa.
Reassortment is of great importance in influenza virus evolution and epidemiology, as it is the only currently known process through which pandemic influenza viruses arise.
Influenza B viruses have been noted for their propensity for reassortment between two co-circulating lineages, called Victoria and Yamagata, in humans, but little has been done to formalize the observed patterns of reassortment.
We apply tools traditionally used in population genetics to quantify gene flow between labeled phylogenies, detect co-assorting alleles on different segments and exploit the properties of time scaled phylogenies to calculate the temporal discordance between lineages being combined by reassortment.
Our primary finding is that out of the 8 genomic segments of influenza B viruses, 3 segments exhibit very limited reassortment with respect to each other.
Although all segments initially had a Victoria and a Yamagata lineage, we detect substantial homogenization through reassortment in the influenza B genome since 1990s, with one or the other lineage being fixed in 5 segments in the influenza B population.
The 3 segments maintaining both Victoria and Yamagata lineages are segments coding for components of the influenza B polymerase: PB1 and PB2 and the segment coding for the virus' main surface glycoprotein HA.
We believe that this finding is of great interest to evolutionary biologists, as it bears all the signs of sympatric speciation, namely co-circulation and the presence of differentiated regions of the genome resisting gene flow.
Additionally, current understanding of the molecular biology of influenza viruses does not offer any explanations for why PB1, PB2 and HA segments should be associated, implying a gap in our knowledge of these viruses.
We also think that the methods we have developed will be of benefit to future studies on reticulate evolution.

\end{newlfm}

\end{document}